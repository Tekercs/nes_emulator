\documentclass[]{report}
\usepackage[backend=biber]{biblatex}
\addbibresource{references.tex}
\usepackage[utf8]{inputenc}
\usepackage[T1]{fontenc}
\usepackage{graphicx}
\graphicspath{images}

% Title Page
\title{NES Emulator}
\author{Sebestyén Bence}

\begin{document}
\maketitle

\tableofcontents

\clearpage

\chapter{Introduction}

\paragraph{}
The Nintendo Entertainment System, frequently called NES, is an home gaming console developed by the Japanese company called Nitendo Company, Limited. The hardware itself can be divided up to five big different part. 
\paragraph{ }
The console main chip was manufactured by Richo, which contains the CPU (Central Processing Unit) and the APU (Audio Processing Unit) \cite{CPU}.  
The processor itself is an 8-bit MOS Technology 6502 with a little difference that the decimal mode is not presented. 
\paragraph{ }
The PPU (Pixel Processing Unit), which was also shipped by Richo, is technically a primitive graphics card which is used by the system to colour and render the graphics pixel by pixel to the Television screens.
\paragraph{ }
The Cartrdige which meant to provide the necessary binary code of the games and also the graphics data for the system. Also gave an opportunity to the developers to implement their own cartridge builds and use it to extend the console's capabilities one great example for that is the first The legend of Zelda game. The game's cartridge also contains a battery powered RAM extension for players to save their game state \cite{ZELD}.
\paragraph{ }
Two 8 button, these buttons are up, down, left, right, select, start, A, B, controller provided the interface for user input to the system. The NES controller was the first controller which introduced the single button plus symbol shaped DPAD. Each Nintendo system were brought some revolutionary design idea to the world of gaming console controllers.
\paragraph{ }
The RAM (Random Access Memory) is the central piece of the hardware which not only holds data but 
trough memory mapping it also connect all the other pieces to the CPU. Therefore the developers can control the full hardware behaviours trough with specific read and write operations to certain memory slots.
\paragraph{ }
In this project these core parts of the hardware will be emulated on an x86-64 machines. As a result, provide an application which is able to run those games which were developed for the original NES system.


\section{History}

\paragraph{ }
 This console was the second console of which was created by the Nitnedo Co. Ltd. and the first which was planned to be sold world-wide. \cite{HIST}
The gaming console was first released in japan 1983. The Japanese version were called Famicon and had  a bright red toy-like design unlike the version which was released world wide.
The homeland release were followed by USA and parts of Europe in 1986 and later in 1987 Australia and rest of Europe.
\paragraph{ }
However, these releases were concerning for Nintendo as '83 was also the year of the great video game crash happened in North America.\cite{CRSH}. But it turned out the perfect opportunity for the console. As the hugely saturated gaming console market shrunk down, most of the competitors get bankrupt, Nintendo rebranded their console and released it as Nintendo Entertainment System worldwide.
\paragraph{ }
 It was a huge success not just because the competitors fall out but also due to two important principle which Nintendo following since then. One of them is unlike other consoles before only certified games could be released to the platform, which meant degradation in quantity and increase in the quality of the games. The other principle was that the actual hardware built from not bleeding edge components therefore making it cheap, more easily accessible.


\chapter{Project Goal}


\chapter{Project Design}

\chapter{Specification and implementation}

\section{RAM}

\subsection{Specification}

\subsection{Implementation}

\subsection{Testing}

\section{Cartridge (ROM)}

\subsection{Specification}

\subsection{Implementation}

\subsection{Testing}

\section{CPU}

\subsection{Specification}

\subsection{Implementation}

\subsection{Testing}

\section{PPU}

\subsection{Specification}

\subsection{Implementation}

\subsection{Testing}

\chapter{Evaluation}

\section{Performance and Precision}

\section{Game Performance}

\chapter{Conclusions}

\printbibliography

\end{document}          